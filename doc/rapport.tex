\documentclass{article}

\usepackage{xcolor}
\usepackage[T1]{fontenc}
\usepackage{palatino}
\usepackage{courier}
\usepackage{alltt}
\usepackage{longtable}
\DeclareTextSymbol{\QT}{T1}{39}
\DeclareTextSymbol{\COMMA}{T1}{44}
\DeclareTextSymbol{\COLON}{T1}{58}
\DeclareTextSymbol{\SC}{T1}{59}
\DeclareTextSymbol{\BS}{T1}{92}
\DeclareTextSymbol{\CI}{T1}{94}
\DeclareTextSymbol{\TI}{T1}{126}
\definecolor{navy}{rgb}{0.15, 0.15, 0.45} 
\definecolor{myblue}{rgb}{0.25, 0.25, 0.645} 
\definecolor{darkred}{rgb}{0.845, 0.125, 0.125} 
\definecolor{grey}{rgb}{0.4, 0.4, 0.4} 
\definecolor{darkgreen}{rgb}{0.125, 0.845, 0.125} 
\definecolor{leaf}{rgb}{0.1, 0.9, 0.1} 

\newcommand{\mlkeywordA}[1]{\mbox{\color{cyan}{\textbf{\texttt{#1}}}}}
\newcommand{\mlkeywordB}[1]{\mbox{\color{navy}{\textbf{\texttt{#1}}}}}
\newcommand{\mlkeyword}[1]{\mbox{\color{red}{#1}}}
\newcommand{\mloperator}[1]{\mbox{\color{darkgreen}{#1}}}
\newcommand{\mlmodulename}[1]{\mbox{\color{navy}{#1}}}
\newcommand{\mlstring}[1]{\mbox{\color{navy}{#1}}}
\newcommand{\mlcomments}[1]{\mbox{\color{grey}{#1}}}
\newcommand{\mlcodeline}[2]{\tiny\sl #1 & \begin{minipage}[c]{0.8\linewidth}\begin{alltt}\mbox{#2}\end{alltt}\end{minipage}\\}
% ---------------------------------------------------------------------

\usepackage{graphicx,enumerate,calc,ifthen}
\usepackage[french]{babel}
\usepackage[utf8]{inputenc}
\usepackage{float}
\floatstyle{boxed} 
\restylefloat{figure}

%%%%%%%%%% Start TeXmacs macros
\newcommand{\nocomma}{}
\newcommand{\tmem}[1]{{\em #1\/}}
\newcommand{\tmname}[1]{\textsc{#1}}
\newcommand{\tmsamp}[1]{\textsf{#1}}
\newcommand{\tmstrong}[1]{\textbf{#1}}
\newcommand{\tmtextbf}[1]{{\bfseries{#1}}}
\newcommand{\tmtextit}[1]{{\itshape{#1}}}
\newcommand{\tmtextmd}[1]{{\mdseries{#1}}}
\newenvironment{enumeratealpha}{\begin{enumerate}[a{\textup{)}}] }{\end{enumerate}}
\newenvironment{itemizedot}{\begin{itemize} \renewcommand{\labelitemi}{$\bullet$}\renewcommand{\labelitemii}{$\bullet$}\renewcommand{\labelitemiii}{$\bullet$}\renewcommand{\labelitemiv}{$\bullet$}}{\end{itemize}}
\newcommand{\tmfloatcontents}{}
\newlength{\tmfloatwidth}
\newcommand{\tmfloat}[5]{
  \renewcommand{\tmfloatcontents}{#4}
  \setlength{\tmfloatwidth}{\widthof{\tmfloatcontents}+1in}
  \ifthenelse{\equal{#2}{small}}
    {\ifthenelse{\lengthtest{\tmfloatwidth > \linewidth}}
      {\setlength{\tmfloatwidth}{\linewidth}}{}}
    {\setlength{\tmfloatwidth}{\linewidth}}
  \begin{minipage}[#1]{\tmfloatwidth}
    \begin{center}
      \tmfloatcontents
      \captionof{#3}{#5}
    \end{center}
  \end{minipage}}
%%%%%%%%%% End TeXmacs macros

\begin{document}
\tableofcontents
\section{Introduction}
\section{Le jeu othello}
\subsection{Principes du jeu}
\section{L'intelligence artificielle}
\subsection{Méthode aléatoire}
\subsection{Méthode du minimax}
\subsubsection{Algorithme du Minimax}
\subsubsection{Problème d'espace et de temps de recherche}
\subsubsection{Le Minimax à profondeur limitée}
\subsubsection{Le Minimax avec élagage {\tmem{Alpha Beta}}}

\section{Implémentation}

Dans cette partie nous allons détailler les étapes d'implémentation de l'application. 

\subsection{Compilation}

Afin de générer les fichiers binaires, nous avons créé un Makefile très générique paramétrable à souhait. Il semblerait que les performances d'exécution soit optimisées en utilisant le compilateur ocamlopt. Ce qui rend le choix intéressant  dans un cas d'utilisation comme le nôtre ou le programme est sujet à des explosions combinatoires. 

Pour une compilation standard avec \texttt{ocamlc}, utiliser : \texttt{make}

\begin{figure}[H]
\caption{Utilisation du Makefile}
\begin{verbatim}
# Pour recompiler le système progressivement :
#     make
# Pour recalculer les dépendances entre les modules :
#     make depend
# Pour supprimer l'exécutable et les fichiers compilés :
#     make clean
# Pour compiler avec le compileur de code natif
#     make opt
\end{verbatim}
\end{figure}

\subsection{Utilisation}

Afin de lancer l'application, tapez: \texttt{./othello}

Il est possible de spécifier des arguments à l'exécutable qui vont influencer le déroulement du jeu. Vous pouvez obtenir la liste ces options avec :

\begin{figure}[H]
\caption{Utilisation de l'exécutable}
\begin{verbatim}
$ ./othello --help
othello
  -size <int> : Taille d'une case en pixels
  -ia <bool> : Intelligence artificielle on/off
               | true  -> Minimax ab 
               | false -> Case aléatoire
  -depth <int> : Profondeur maximale d'exploration 
                 de l'arbre des possibilités Minimax 
  -background <int> <int> <int> : Couleur du fond (RGB)
  -help : Afficher cette liste d'options
  --help : Afficher cette liste d'options
\end{verbatim}
\end{figure}

Les options par défaut de l'application sont :

\begin{figure}[H]
\caption{Configuration par défaut}
{\scriptsize\noindent
\begin{longtable}{r|l}
\mlcodeline{1}{\mlkeyword{val}~cell\_{}size~\mloperator{\mbox{\COLON}}~int~ref~\mlkeyword{=}~\mloperator{\{}contents~\mlkeyword{=}~50\mloperator{\}}
}
\mlcodeline{2}{\mlkeyword{val}~bg\_{}r~\mloperator{\mbox{\COLON}}~int~ref~\mlkeyword{=}~\mloperator{\{}contents~\mlkeyword{=}~50\mloperator{\}}
}
\mlcodeline{3}{\mlkeyword{val}~bg\_{}g~\mloperator{\mbox{\COLON}}~int~ref~\mlkeyword{=}~\mloperator{\{}contents~\mlkeyword{=}~150\mloperator{\}}
}
\mlcodeline{4}{\mlkeyword{val}~bg\_{}b~\mloperator{\mbox{\COLON}}~int~ref~\mlkeyword{=}~\mloperator{\{}contents~\mlkeyword{=}~50\mloperator{\}}
}
\mlcodeline{5}{\mlkeyword{val}~size~\mloperator{\mbox{\COLON}}~int~ref~\mlkeyword{=}~\mloperator{\{}contents~\mlkeyword{=}~8\mloperator{\}}
}
\mlcodeline{6}{\mlkeyword{val}~ia~\mloperator{\mbox{\COLON}}~bool~ref~\mlkeyword{=}~\mloperator{\{}contents~\mlkeyword{=}~\mlkeywordB{true}\mloperator{\}}
}
\mlcodeline{7}{\mlkeyword{val}~depth~\mloperator{\mbox{\COLON}}~int~ref~\mlkeyword{=}~\mloperator{\{}contents~\mlkeyword{=}~4\mloperator{\}}
}
\end{longtable}
}
\end{figure}

\subsection{Liste exhaustive des prototypes}


\subsection{Algorithmes intéressants}

\subsubsection{Direction Légale}

Methode de test de direction légal => Vrai si la direction est légale

\begin{figure}[H]
\caption{Direction légale}
{\scriptsize\noindent
\begin{longtable}{r|l}
\mlcodeline{1}{\mlcomments{(*~Methode~de~test~de~direction~légal~{=>\mbox{}}~Vrai~si~la~direction~est~légale~*)}
}
\mlcodeline{2}{\mlkeywordA{let}~playable\_{}dir~board~c~(x\mloperator{\mbox{,}}~y)~(dx\mloperator{\mbox{,}}~dy)~\mlkeyword{=}
}
\mlcodeline{3}{~~\mlkeywordA{let~rec}~playable\_{}dir\_{}rec~(x\mloperator{\mbox{,}}~y)~valid~\mlkeyword{=}
}
\mlcodeline{4}{~~~~\mlkeyword{if}~not~(check\_{}pos~board~x~y)~\mlkeyword{then}
}
\mlcodeline{5}{~~~~~~\mlkeywordB{false}
}
\mlcodeline{6}{~~~~\mlkeyword{else}~(
}
\mlcodeline{7}{~~~~~~\mlkeyword{match}~board\mloperator{.}(x)\mloperator{.}(y)~\mlkeyword{with}
}
\mlcodeline{8}{~~~~~~~~\mloperator{|}~Empty~\mlkeyword{->}~\mlkeywordB{false}
}
\mlcodeline{9}{~~~~~~~~\mloperator{|}~cell~\mlkeyword{->}
}
\mlcodeline{10}{~~~~~~~~~~\mlkeyword{if}~cell~\mlkeyword{=}~(get\_{}opponent~c)~\mlkeyword{then}
}
\mlcodeline{11}{~~~~~~~~~~~~playable\_{}dir\_{}rec~(x~\mloperator{+}~dx\mloperator{\mbox{,}}~y~\mloperator{+}~dy)~\mlkeywordB{true}
}
\mlcodeline{12}{~~~~~~~~~~\mlkeyword{else}
}
\mlcodeline{13}{~~~~~~~~~~~~valid
}
\mlcodeline{14}{~~~~)
}
\mlcodeline{15}{~~~~\mlkeywordA{in}~playable\_{}dir\_{}rec~(x~\mloperator{+}~dx\mloperator{\mbox{,}}~y~\mloperator{+}~dy)~\mlkeywordB{false}
}
\mlcodeline{16}{\mloperator{\mbox{\SC}\mbox{\SC}}}
\end{longtable}
}
\end{figure}


\subsubsection{Coup Légal}

\begin{figure}[H]
\caption{Coup légal}
{\scriptsize\noindent
\begin{longtable}{r|l}
\mlcodeline{1}{\mlcomments{(*~Methode~de~test~de~coup~légal~{=>\mbox{}}~Vrai~si~le~coup~est~légal~*)}
}
\mlcodeline{2}{\mlkeywordA{let}~playable\_{}cell~board~c~x~y~\mlkeyword{=}
}
\mlcodeline{3}{~~\mlkeyword{if}~not~(check\_{}pos~board~x~y)~\mlkeyword{then}
}
\mlcodeline{4}{~~~~\mlkeywordB{false}
}
\mlcodeline{5}{~~\mlkeyword{else}~(
}
\mlcodeline{6}{~~~~\mlkeywordA{let}~directions~\mlkeyword{=}~\mloperator{[}~
}
\mlcodeline{7}{~~~~~~\mloperator{(-}1\mloperator{\mbox{,}}~-1)\mloperator{\mbox{\SC}}~\mloperator{(-}1\mloperator{\mbox{,}}~0)\mloperator{\mbox{\SC}}~\mloperator{(-}1\mloperator{\mbox{,}}~1)\mloperator{\mbox{\SC}}~
}
\mlcodeline{8}{~~~~~~(0~\mloperator{\mbox{,}}~-1)\mloperator{\mbox{\SC}}~\mlcomments{(*~X~*)}~~(0~\mloperator{\mbox{,}}~1)\mloperator{\mbox{\SC}}~
}
\mlcodeline{9}{~~~~~~(1~\mloperator{\mbox{,}}~-1)\mloperator{\mbox{\SC}}~(1~\mloperator{\mbox{,}}~0)\mloperator{\mbox{\SC}}~(1~\mloperator{\mbox{,}}~1)~
}
\mlcodeline{10}{~~~~\mloperator{]}
}
\mlcodeline{11}{~~~~\mlkeywordA{in}~\mlkeyword{match}~board\mloperator{.}(x)\mloperator{.}(y)~\mlkeyword{with}
}
\mlcodeline{12}{~~~~~~\mloperator{|}~Empty~\mlkeyword{->}~(~\mlkeywordB{true}~\mloperator{\&\&}~(	
}
\mlcodeline{13}{		~~~~\mlmodulename{List}\mbox{}\mloperator{.}fold\_{}left
}
\mlcodeline{14}{		~~~~~~(\mlkeyword{fun}~a~b~\mlkeyword{->}~a~\mloperator{||}~b)~
}
\mlcodeline{15}{		~~~~~~\mlkeywordB{false}
}
\mlcodeline{16}{~~~~~~~~~~~~~~(\mlmodulename{List}\mbox{}\mloperator{.}map
}
\mlcodeline{17}{~~~~~~~~~~~~~~~~(\mlkeyword{fun}~d~\mlkeyword{->}~playable\_{}dir~board~c~(x\mloperator{\mbox{,}}~y)~d)
}
\mlcodeline{18}{~~~~~~~~~~~~~~~~directions	
}
\mlcodeline{19}{~~~~~~~~~~~~~~)
}
\mlcodeline{20}{~~~~~~~~~~~~)
}
\mlcodeline{21}{~~~~~~~~~~)~
}
\mlcodeline{22}{~~~~~~~\mloperator{|}~\mloperator{\_}~\mlkeyword{->}~\mlkeywordB{false}	
}
\mlcodeline{23}{~~)
}
\mlcodeline{24}{\mloperator{\mbox{\SC}\mbox{\SC}}}
\end{longtable}
}
\end{figure}

\subsubsection{Simulation de jeu}

\begin{figure}[H]
\caption{Simuler un coup}
{\scriptsize\noindent
\begin{longtable}{r|l}
\mlcodeline{1}{\mlcomments{(*~Méthode~pour~simuler~le~jeu~sur~une~case~*)}
}
\mlcodeline{2}{\mlkeywordA{let}~sim\_{}play\_{}cell~board~c~x~y~\mlkeyword{=}
}
\mlcodeline{3}{~~\mlkeywordA{let}~sim\_{}board~\mlkeyword{=}~(copy\_{}board~board)~\mlkeywordA{in}
}
\mlcodeline{4}{~~\mlkeywordA{let}~directions~\mlkeyword{=}~\mloperator{[}~
}
\mlcodeline{5}{~~~~\mloperator{(-}1\mloperator{\mbox{,}}~-1)\mloperator{\mbox{\SC}}~\mloperator{(-}1\mloperator{\mbox{,}}~0)\mloperator{\mbox{\SC}}~\mloperator{(-}1\mloperator{\mbox{,}}~1)\mloperator{\mbox{\SC}}~
}
\mlcodeline{6}{~~~~(0~\mloperator{\mbox{,}}~-1)\mloperator{\mbox{\SC}}~\mlcomments{(*~X~*)}~~(0~\mloperator{\mbox{,}}~1)\mloperator{\mbox{\SC}}~
}
\mlcodeline{7}{~~~~(1~\mloperator{\mbox{,}}~-1)\mloperator{\mbox{\SC}}~(1~\mloperator{\mbox{,}}~0)\mloperator{\mbox{\SC}}~(1~\mloperator{\mbox{,}}~1)~
}
\mlcodeline{8}{~~\mloperator{]}
}
\mlcodeline{9}{~~\mlkeywordA{and}~opponent~\mlkeyword{=}~(get\_{}opponent~c)~
}
\mlcodeline{10}{~~\mlkeywordA{in}~(
}
\mlcodeline{11}{~~~~\mlmodulename{List}\mbox{}\mloperator{.}iter~											
}
\mlcodeline{12}{~~~~~~(\mlkeyword{fun}~(dx\mloperator{\mbox{,}}~dy)~\mlkeyword{->}
}
\mlcodeline{13}{~~~~~~~~\mlkeyword{if}~(playable\_{}dir~sim\_{}board~c~(x\mloperator{\mbox{,}}~y)~(dx\mloperator{\mbox{,}}~dy))~\mlkeyword{then}
}
\mlcodeline{14}{~~~~~~~~~~\mlkeywordA{let~rec}~take~(x\mloperator{\mbox{,}}~y)~\mlkeyword{=}
}
\mlcodeline{15}{~~~~~~~~~~~~\mlkeyword{if}~(check\_{}pos~sim\_{}board~x~y)~\mlkeyword{then}	
}
\mlcodeline{16}{~~~~~~~~~~~~\mlkeyword{if}~(sim\_{}board\mloperator{.}(x)\mloperator{.}(y)~\mlkeyword{=}~opponent)~\mlkeyword{then}~(
}
\mlcodeline{17}{~~~~~~~~~~~~~~sim\_{}board\mloperator{.}(x)\mloperator{.}(y)~\mloperator{<\mbox{}-}~c\mloperator{\mbox{\SC}}
}
\mlcodeline{18}{~~~~~~~~~~~~~~take~(x~\mloperator{+}~dx\mloperator{\mbox{,}}~y~\mloperator{+}~dy)
}
\mlcodeline{19}{~~~~~~~~~~~~)
}
\mlcodeline{20}{~~~~~~~~~~\mlkeywordA{in}~take~(x~\mloperator{+}~dx\mloperator{\mbox{,}}~y~\mloperator{+}~dy)
}
\mlcodeline{21}{~~~~~~)
}
\mlcodeline{22}{~~~~~~directions
}
\mlcodeline{23}{~~)\mloperator{\mbox{\SC}}~
}
\mlcodeline{24}{~~sim\_{}board\mloperator{.}(x)\mloperator{.}(y)~\mloperator{<\mbox{}-}~c\mloperator{\mbox{\SC}}
}
\mlcodeline{25}{~~sim\_{}board
}
\mlcodeline{26}{\mloperator{\mbox{\SC}\mbox{\SC}}}
\end{longtable}
}
\end{figure}

\end{document}
